\chapter{Statistical learning}

\section*{Background}
These are my notes in the course Statistical learning given by Stanford university.

\section*{Chapter 2 - Statistical learning}

Notation: $Y = f(X) + \epsilon$\\

Understand which components of $X = (X_1, X_2, \dots, X_p)$ are important in explaining $Y$.\\

What is a good value for $f(X)$?\\
\begin{equation}
f(x) = \E[Y | X=x]
\end{equation}
(regression function)\\

Ideal/optimal predictor of $Y$.\\
$f(x)=\E[Y|X=x]$ is function that minimizes MSE
\begin{equation*}
\E[(Y-g(X))^2 | X=x],
\end{equation*}
over all functions $g$ at all points $X=x$.\\

$\epsilon = Y - f(X)$, irreducible error
\begin{equation}
\E[(Y-\hat{f}(X))^2 | X=x] = \underbrace{\big(f(x)-\hat{f}(x)\big)^2}_\text{reducible} - \underbrace{\Var(\epsilon)}_\text{irreducible}
\end{equation}\\

Nearest neighbor good for small $p$, $p<4$, and large $N$.\\
NN lousy for large $p$ due to dimensionality.\\
We deal with dimensionality through structured models.\\

\subsection*{Parametric \& structured models}
Linear model is parametric model:
\begin{equation}
f(X)=\beta_0+\beta_1X_1+\dots+\beta_pX_p
\end{equation}\\
Estimate parameters by fitting model to training data.\\
Almost never correct but good approximation.\\
Splines is a sort of smoothing.\\
When tuning a spline so that there are no errors $\epsilon$ on training data the model is \underline{overfitted}. We are overfitting the training data.\\

\subsection*{Trade-offs}
Prediction accuracy vs. interpretability\\
- Linear models easy to interpret, thin-plate splines are not\\
Good fit vs. over-fit or under-fit\\
- When is the fit right?\\
Parsimony vs. black-box\\
- Less or more\\



\section*{Chapter 3 - Linear regression}

\section*{Chapter 4 - Classification}

\section*{Chapter 5 - Resampling}



%From the analysis the following was concluded
%\begin{itemize}
%\item Average score has declined during the year explaining \underline{approximately 20\% of the increase} in unpaid rates.
%\item Remaining 80\% difficult to explain. No explanation could be found by analyzing factors such as; estore group, has paid, acquiring source, scorecard, AOV's, etc.
%\end{itemize}
%
%\section*{Analysis}
%
%In the following section is the deep-dive analysis presented. The hypotheses explored are presented and discussed and conclusions are drawn in cases there is any conclusions to be drawn.
%
%\subsection*{Hypotheses}
%Several hypotheses were tried out during the analysis and are stated below with comments. See Appendix~\ref{sec:appendixFigures} for a list of figures supporting the conclusions drawn from the analysis.\\
%
%\textbf{Acquiring source}\\
%The increase is present within all acquiring sources, see Fig.~\ref{fig:rate30_dev_as}-\ref{fig:pd_dev_as}. However, the PD rate for KP seems to be stable over time.\\
%
%\textbf{Merchant specific, estore group \& estore segment}\\
%See acquiring source hypothesis. The increase is not driven by a specific merchant, estore segment or estore group, see Fig.~\ref{fig:rate30_dev_mer}-\ref{fig:pd_dev_mer}. The increase is present within all of these sub-segments.\\
%
%\textbf{Score distribution}\\
%The score distribution is successively skewed to the left during the year, i.e. the population entering entails a higher risk, see Fig.~\ref{fig:score_dev_1}. The general decline in average score seems to be present for returning customers, Fig.~\ref{fig:score_dev_ret}, and not for new customers, Fig.~\ref{fig:score_dev_new}. In Table~\ref{tab:rate_dev_qr} are the unpaid rates for the two first quarters of 2018. We see an increase of 45-55\% in unpaid rates but only a increase of 8\% in PD rate (which is directly attributable to the score distribution), hence the increase in unpaid rates can be explained by the PD rate increase by approximately 20\%. That leaves us with 80\% of the increase unexplained.
%
%\begin{table}[!ht]
%\renewcommand{\arraystretch}{1.2}
% \caption{Unpaid rates per quarter. The increase in PD rate can explain }
% \label{tab:rate_dev_qr}
% \centering
% \begin{tabularx}{\textwidth}{XXXXX}\thickhline
%Quarter & Unpaid rate 30 & Unpaid rate 60 & Unpaid rate 90 & PD rate \\ \thickhline
%2018Q1 & 0.0913 & 0.0316 & 0.0211 & 0.00992 \\
%2018Q2 & 0.132 & 0.0491 & 0.0305 & 0.0107 \\
%2018Q3 & 0.0850 & 0.0154 & 0.000430 & 0.0120 \\
%Increase (\%)&&&&\\
%Q1-Q2 & 45\% & 55\% & 45\% & 8\% \\
% \thickhline
% \end{tabularx}
%\end{table}
%
%\textbf{New vs. returning customers}\\
%The increase is present within all unpaid rates for both new and returning customers, see Fig.~\ref{fig:rate60_dev_hp}. However, looking at the PD rate it seems to be a more clear upward trend for returning customers rather than for new. The sharp decline in volume/transactions is mainly driven by returning customers and could explain why the PD rate increase.\\
%
%\textbf{Different AOV intervals}\\
%We doesn't see any difference between different AOV bins. The increase is present within all AOV bins, see Fig.~\ref{fig:rate60_dev_aov}-\ref{fig:pd_dev_aov}.\\
%
%\textbf{Creation vs. activation date}\\
%Vast majority of data has same creation as activation date. Nothing that drives the increase.\\
%
%\textbf{Scorecard to use}\\
%The increase is present within all different scorecards. No conclusions are drawn from investigating this.\\
%
%\textbf{Mix effect merchant dimension}\\
%Investigated if there could be mix effects driving the increase. But looking at Table.~\ref{tab:mix_mer} is clear that there is a minor increase due to mix effects but its barley notable.\\
%
%\begin{table}[!ht]
%\renewcommand{\arraystretch}{1.2}
% \caption{Unpaid rate 90 and volume share for each quarter for 10 biggest merchants and rest. The unpaid rates are then weighted with their respective volume share and a total unpaid rate is calculated. By computing unpaid rate using Q1 volume share (unpQ2 adjusted) and Q2 volume share (UnpaidQ2) we get at sens if there is any mix effects present. Unfortunately there does not seem to be any mix effects influencing the unpaid rates increase.}
% \label{tab:mix_mer}
% \centering
% \begin{tabularx}{\textwidth}{XXXXX}\thickhline
%& Unpaid rate 90 & & Share volume &\\
%Merchant & 2018Q1 & 2018Q2 & 2018Q1 & 2018Q2\\ \thickhline
%jdsports & 4.97\% & 7.05\% & 1.66\% & 2.30\%\\
%lesara & 2.60\% & 3.32\% & 1.72\% & 1.71\%\\
%bonaparteshop & 0.12\% & 0.20\% & 2.75\% & 2.71\%\\
%missetam & 0.73\% & 1.14\% & 2.79\% & 3.59\%\\
%nike & 1.57\% & 2.50\% & 3.43\% & 2.58\%\\
%shop.bestseller & 0.68\% & 1.07\% & 4.95\% & 5.34\%\\
%mediamarkt & 2.99\% & 6.27\% & 5.22\% & 2.86\%\\
%Wish & 9.16\% & 6.18\% & 5.54\% & 5.28\%\\
%omoda & 0.64\% & 0.89\% & 9.27\% & 10.40\%\\
%zara & 1.02\% & 1.45\% & 15.18\% & 15.90\%\\
%Other & 2.08\% & 3.02\% & 47.48\% & 47.33\%\\ \thickhline
%&UnpQ1&UnpQ2 adj.&UnpQ2&\\
%&2.11\%&2.76\%&2.64\%&\\
% \thickhline
% \end{tabularx}
%\end{table}
%
%\textbf{Unpaid 90 vs. PD}\\
%Looked into the ratio between Unpaid rate 90 and PD rate. No additional conclusions could be drawn, see Fig.~\ref{fig:unp90pd_dev}-\ref{fig:unp90pd_dev_as}.\\
%
%\textbf{Exclude churned merchants}\\
%Some merchants in NL churned during 2018 and it was investigated if these merchants was driver of increase. The increase is present even when these estore id's are excluded and hence these merchants does not drive the increase.\\