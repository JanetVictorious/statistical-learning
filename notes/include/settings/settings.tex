% CREATED BY DAVID FRISK, 2014
\usepackage[eng,exjobb]{include/frontmatter/KTHEEtitlepage}
\usepackage{datetime}
% BASIC SETTINGS
\usepackage{moreverb}								% List settings
\usepackage{textcomp}								% Fonts, symbols etc.
\usepackage{lmodern}								% Latin modern font
\usepackage{helvet}									% Enables font switching
\usepackage[T1]{fontenc}							% Output settings
\usepackage[english]{babel}							% Language settings
\usepackage[utf8]{inputenc}							% Input settings
\usepackage{array}
\usepackage{ragged2e}
\newcolumntype{P}[1]{>{\RaggedRight\hspace{0pt}}p{#1}}
\usepackage{amsmath}								% Mathematical expressions (American mathematical society)
\usepackage{amssymb}								% Mathematical symbols (American mathematical society)
\usepackage{graphicx}								% Figures
\usepackage{subfig}									% Enables subfigures
\numberwithin{equation}{chapter}					% Numbering order for equations
\numberwithin{figure}{chapter}						% Numbering order for figures
\numberwithin{table}{chapter}						% Numbering order for tables
\usepackage{listings}								% Enables source code listings
\usepackage{chemfig}								% Chemical structures
\usepackage[top=3cm, bottom=3cm,
			inner=3cm, outer=3cm]{geometry}			% Page margin lengths			
\usepackage{eso-pic}								% Create cover page background
%\usepackage{geometry}  
\usepackage{tabu}
\usepackage{tabulary}
\usepackage{tabularx}
\usepackage{nomencl}
\usepackage{glossaries}
\usepackage{chngpage}
\usepackage{framed}
\usepackage{multirow}
\usepackage{titling}
\usepackage{soul}
\usepackage{float}
\usepackage{longtable}
\usepackage{pdfpages}
%\usepackage{libertine} 
% OPTIONAL SETTINGS (DELETE OR COMMENT TO SUPRESS)

% Disable automatic indentation (equal to using \noindent)
\setlength{\parindent}{0cm}                         


% Caption settings
% \usepackage[labelfont=bf, textfont=it]{caption} 
% This caption setting is similar to IEEE articles
\usepackage[font=small, labelsep=period]{caption}
\captionsetup[figure]{name=Fig.}

		  	
% Activate clickable links in table of contents  	
\usepackage{hyperref}								
\hypersetup{colorlinks, citecolor=black,
   		 	filecolor=black, linkcolor=black,
    		urlcolor=black}


% Define the number of section levels to be included in the t.o.c. and numbered	(3 is default)	
\setcounter{tocdepth}{5}							
\setcounter{secnumdepth}{5}	

\usepackage{gensymb}

% Chapter title settings
\usepackage{titlesec}		
\titleformat{\chapter}[display]{\huge\bfseries}{}{0ex}{}[\rule{\textwidth}{2pt}]


% Header and footer settings (Select TWOSIDE or ONESIDE layout below)
\usepackage{fancyhdr}								
\pagestyle{fancy}

% Select one-sided (1) or two-sided (2) page numbering
\def\layout{2}	% Choose 1 for one-sided or 2 for two-sided layout
% Conditional expression based on the layout choice
\fancyhf{}
\rhead{\nouppercase{\leftmark}}
\fancyfoot[C]{\thepage}
\renewcommand{\headrulewidth}{2pt}

\makeatletter
\newcommand*{\centerfloat}{%
  \parindent \z@
  \leftskip \z@ \@plus 1fil \@minus \textwidth
  \rightskip\leftskip
  \parfillskip \z@skip}
\makeatother

% Create thick horizontal lines used in tables
\makeatletter
\newcommand{\thickhline}{%
    \noalign {\ifnum 0=`}\fi \hrule height 1pt
    \futurelet \reserved@a \@xhline
}
\newcolumntype{"}{@{\hskip\tabcolsep\vrule width 1pt\hskip\tabcolsep}}
\makeatother

\newcommand\theblurb{}
\newcommand\blurb[1]{\renewcommand\theblurb{#1}}
\newcommand\thetrita{}
\newcommand\trita[1]{\renewcommand\thetrita{#1}}
\newcommand\theforeigntitle{}
\newcommand\foreigntitle[1]{\renewcommand\theforeigntitle{#1}}
\newcommand\theexaminer{}
\newcommand\examiner[1]{\renewcommand\theexaminer{#1}}
\newcommand\thesupervisor{}
\newcommand\supervisor[1]{\renewcommand\thesupervisor{#1}}
\newcommand\theapprovaldate{}
\newcommand\approvaldate[1]{\renewcommand\theapprovaldate{#1}}
\newcommand\thecompany{}
\newcommand\company[1]{\renewcommand\thecompany{#1}}
\newcommand\thecompanycontact{}
\newcommand\companycontact[1]{\renewcommand\thecompanycontact{#1}}

\newcommand{\nomunit}[1]{\renewcommand{\nomentryend}{\hspace*{\fill}#1}}

\newcommand*{\captionsource}[2]{%
\centering
  \caption[{#1}]{%
    #1%
    \\\hspace{\linewidth}%
    \textbf{Source:} #2%
  }%
}

\let\oldtabular\tabular
\renewcommand{\tabular}{\footnotesize\oldtabular}


% Tikz library settings
\usepackage{tikz}
\usetikzlibrary{shapes,arrows,patterns}

% Taken from latex beamer template from Pedro Lima at KTH
\DeclareMathOperator*{\argmin}{argmin}
\DeclareMathOperator*{\argmax}{argmax}
\DeclareMathOperator*{\diag}{diag}
\DeclareMathOperator*{\Erf}{Erf}
\DeclareMathOperator*{\Erfinv}{Erfinv}
\DeclareMathOperator*{\E}{E}
\DeclareMathOperator*{\p}{P}
\DeclareMathOperator*{\Var}{Var}
\DeclareMathOperator*{\Std}{Std}
\DeclareMathOperator*{\VaR}{VaR}
\DeclareMathOperator*{\dom}{dom}
\DeclareMathOperator*{\dist}{dist}
\DeclareMathOperator*{\sgn}{sgn}
\DeclareMathOperator*{\spec}{spec}
